%~~~~~~~~~~~~~~~~~~~~~~~~~~~~~~~~~~~~~~~~~~~~~~~~~~~~~~~~~~~~~~~~~~~~~
%    File      : ptrab_expectativas
%~~~~~~~~~~~~~~~~~~~~~~~~~~~~~~~~~~~~~~~~~~~~~~~~~~~~~~~~~~~~~~~~~~~~~

As demandas para a \siglacntrt\ estabelecidas pela Lei n\ele\ 8.854/1994 criam um painel de necessidades de informação e serviços sensitivos, altamente técnicos, abrangentes em diversas áreas (ciência da informação, engenharia, comunicação entre outras) e precisos. Essas características demonstram um ecossistema de informação dinâmico, rápido e de grande volume, com uma nuvem de dados e um fluxo de sinais de grande complexidade.

Esse cenário caracteriza justamente o tipo de ambiente onde o \cpai\ atua com excelência: uma complexa rede de dados e informações, que precisam ser definidos, tratados e organizados em uma arquitetura da informação que dê sentido e direcionamento à informação, criando arquitetura da informação que auxilie a \siglacntrt\ a tirar o máximo proveito da imensa malha de dados a que tem acesso.

Sob a perspectiva da relevância, o projeto é de grande interesse, apelo e potencial para a inovação ao aliar um setor de atividade estratégica para a soberania nacional com o potencial transformador da Arquitetura da Informação, instrumentalizando a \siglacntrt\ para o uso de soluções tecnológicas capazes de modelar processos, serviços e demandas de forma racional e objetiva, maximizando os resultados dos esforços técnicos da organização.
